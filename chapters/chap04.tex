\chapter{Ontologie per Energy Storage System}

L'obiettivo principale delle ontologie per \textit{Energy Storage System} (ESS)
è di fornire una descrizione
semantica standardizzata dei dati relativi agli ESS, in modo da consentire
l'interoperabilità tra i sistemi di gestione dell'energia e i dispositivi ESS
provenienti da diversi fornitori. Questa interoperabilità è essenziale per
garantire l'efficienza, l'affidabilità e la sicurezza delle reti elettriche
intelligenti (smart grid) e per consentire l'integrazione di fonti di energia
rinnovabile.

Le ontologie ESS comprendono concetti come la capacità di stoccaggio
dell'energia, la capacità di erogazione di potenza, la capacità di ricarica, il
livello di sicurezza e
molti altri. Questi concetti sono organizzati in una struttura gerarchica che
definisce le relazioni tra di essi.

Esistono molte ontologie che coprono questi aspetti e ne estendono i concetti,
tra cui: \textit{OntoPowSys}, \textit{Energy Storage Systems Ontology},
\textit{Open Energy Ontology} e l'ecosistema basato su \textit{EMMO}.

\section{OntoPowSys}
OntoPowSys \cite{OntoPowSys} è un'ontologia del dominio dei sistemi di
alimentazione, che è stata
sviluppata con lo scopo di supportare l'analisi e la progettazione di sistemi
di alimentazione elettrica, in particolare per applicazioni industriali e di
produzione di energia rinnovabile.

Si basa su una struttura a tre livelli, che comprende:
la descrizione dei componenti, la descrizione delle funzioni e la descrizione
delle proprietà.

L'ontologia è stata utilizzata per sviluppare strumenti di supporto alla
decisione per l'analisi e la progettazione dei sistemi di alimentazione
elettrica, ad esempio per la selezione e la configurazione dei componenti del
sistema. Inoltre, OntoPowSys è stato integrato con altre ontologie del dominio
dell'energia, come la Open Energy Ontology, per creare una visione più completa
e integrata del sistema energetico.

Inoltre, estende i concetti di supporto creando un framework per l'esecuzione
di modelli matematici in JPS (Joint Processing System è un sistema informatico
che
utilizza ontologie per integrare conoscenze provenienti da diverse fonti e
consentire l'esecuzione di studi e simulazioni).

\subsection{Casi d'uso}
In seguito verrà presentato qualche esempio di applicazione dell'ontologia
OntoPowSys.

\subsubsection{Optimal Power Flow}
OntoPowSys è stata integrata nell'agente Optimal Power Flow (OPF) della
piattaforma JPS, che consente di risolvere il problema dell'ottimizzazione
della distribuzione di energia elettrica. Esso utilizza l'ontologia
OntoPowSys per acquisire conoscenze sul sistema elettrico e fornire soluzioni
ottimali al problema dell'OPF.

Le informazioni vengono acquisite tramite query
SPARQL sul grafo di conoscenza JPS e utilizzate per eseguire i modelli
matematici. I risultati vengono aggiornati nel grafo di conoscenza e
visualizzati tramite strumenti appositi.

\subsubsection{Bio Diesel Plant Simulation}
Nell'ambito della simulazione di una bio-raffineria, OntoPowSys viene
utilizzata per la modellizzazione delle proprietà dei materiali coinvolti nei
processi di produzione del biodiesel. Questi modelli vengono integrati nella
piattaforma JPS, che utilizza web services e agenti specifici per eseguire
simulazioni e analisi.

In particolare, un agente di JPS viene utilizzato per lo
studio delle interazioni tra le varie fasi del processo di produzione del
biodiesel. Le informazioni raccolte dalla simulazione vengono utilizzate per
aggiornare la conoscenza presente nella piattaforma JPS e per generare
risultati visualizzabili tramite appositi strumenti.

\section{Open Energy Ontology}
Open Energy Ontology \cite{OpenEnergyOntology} (OEO) è un'ontologia di dominio
nel campo della
modellizzazione dei sistemi energetici. Ha lo scopo di standardizzare la
terminologia nel campo e di fornire una struttura logica per l'integrazione dei
dati e l'aggregazione. L'ontologia può essere utilizzata anche per la creazione
di modelli di acquisizione dei dati, la visualizzazione dei dati e l'estrazione
di informazioni dal testo e dai dati.

Gli obiettivi dell'Open Energy Ontology \cite{OEO} sono orientati alla
crescente
sofisticazione e interdipendenza della modellizzazione dei sistemi energetici.
Ciò richiede una maggiore automazione delle interfacce tra modelli e una
comprensione semantica dei dati.

\subsection{Casi d'uso}
In seguito verranno descritti casi d'uso provenienti dai progetti
\textit{SzenarienDB}, \textit{LOD-GEOSS} e \textit{SIROP}.

\subsubsection{Scenario description}
Lo scopo era quello di creare un quadro che permettesse di collegare in modo
più strutturato e trasparente i risultati di ricerche energetiche, le basi di
dati utilizzate e le assunzioni implicite ed esplicite coinvolte.

Da una raccolta di proprietà degli scenari e degli
studi energetici che coprono le informazioni più importanti, proprietà inserite
in un foglio di calcolo da cui è stato creato un
grafo di conoscenza RDF, che utilizza
le classi e le relazioni definite nella OEO sulla base di questi fogli di
calcolo.

La piattaforma Open Energy è stata estesa con una sezione aggiuntiva che
consente agli utenti
di visualizzare ed editare le informazioni memorizzate nel grafo di conoscenza.
Questo permette ai ricercatori di rendere la loro ricerca più disponibile al
pubblico e facilita una descrizione più strutturata del paesaggio di basi di
dati, modelli energetici e risultati scientifici.

\subsubsection{Data representation for the core energy market data register}
Il registro centrale dei dati del mercato energetico (MaStR) pubblicato
dall'Agenzia federale tedesca per le reti energetiche nel 2019 fornisce una
lista completa di tutte le centrali registrate in Germania che si connettono
alla rete. Tuttavia, l'API del MaStR è complessa e presenta limitazioni
nell'accesso ai dati, come ad esempio il numero limitato di richieste
giornaliere per utente.

Per migliorare l'accessibilità ai dati, è stato
sviluppato uno strumento open source per estrarre dati dal MaStR, arricchendoli
con metadati e rendendoli disponibili sulla piattaforma Open Energy Platform,
dove possono essere scaricati senza bisogno di registrazione.

Con l'Ontology
for Energy Systems (OEO) è possibile annotare il dataset MaStR e semplificare
le query complesse attraverso una mappatura dell'API a un endpoint per il
protocollo SPARQL che utilizza i termini definiti nell'ontologia. Ciò permette
una maggiore facilità d'uso per la raccolta di dati complessi e l'arricchimento
del dataset con dipendenze logiche.

\subsubsection{Data annotation of an energy meteorological time series data
    set}

Questo caso d'uso tratta dell'annotazione di serie temporali meteorologiche
energetiche utilizzate come
dati di input o output in modelli di sistemi energetici. L'annotazione coerente
e completa non è banale poiché esistono diverse modalità
di descrizione delle serie temporali e le convenzioni di annotazione variano a
seconda dei domini.

L'OEO facilita questo processo fornendo definizioni concise e non ambigue per i
diversi concetti di annotazione. Ciò consente alla struttura e al contenuto
delle serie temporali meteorologiche energetiche e di altre discipline di
essere descritte completamente e identificate senza sforzo interdisciplinare.

\subsubsection{Interface homogenization of the FINE energy system model
    framework}

Utilizzando l'OEO, si vuole omogeneizzare l'annotazione all'interno di un
database distribuito di inventari di dati e parametri funzionali attesi o
forniti dalle interfacce dei modelli.
Si riduce l'eterogeneità delle descrizioni delle interfacce e si minimizza lo
sforzo dei programmatori e
degli utenti per produrle e comprenderle.

Sulla base della descrizione dell'interfaccia attualmente esistente in cui i
singoli parametri di funzione sono denominati e definiti, si sta attualmente
esplorando in che misura esiste già una copertura con la terminologia dell'OEO
e in che punti è necessario adattare l'interfaccia o l'ontologia. Utilizzando
queste applicazioni specifiche del modello, si mira a sviluppare le migliori
pratiche che possono essere utilizzate per omogeneizzare la connessione dei
dati ai modelli e, alla fine, lo scambio di dati tra i modelli stessi e per
promuovere lo scambio scientifico all'interno della comunità internazionale del
sistema energetico.

\section{Energy Storage Systems Ontology}
L'Energy Storage Systems Ontology \cite{EnergyStorageSystemOntology} è
un'ontologia, che dispone di 10 classi, sviluppata per
rappresentare le conoscenze e le informazioni relative ai sistemi di accumulo
dell'energia (ESS). Questa ontologia è un modulo del “Domain Analysis-Based
Global Energy ontology” e si occupa di rappresentare dispositivi di accumulo di
energia, come batterie o condensatori.

\section{EMMO-based Ecosystem}
Elementary Multiperspective Material Ontology (EMMO) \cite{emmo} è un
framework ontologico per le
scienze applicate e l'ingegneria. È stato progettato per soddisfare le esigenze
di una descrizione semantica profondamente radicata nelle scienze fisiche,
incorporando la descrizione dei materiali da una rigorosa prospettiva fisica,
le relazioni formali tra livelli di granularità per facilitare la descrizione
di materiali multi-scala e la definizione di processi materiali per catturare
il cambiamento ed evoluzione dei materiali come catena di diversi stati.

Attualmente definisce quattro prospettive:
\begin{itemize}
    \item \textbf{Riduzionistica}: introduce una relazione di
          parzialità diretta non transitiva.
    \item \textbf{Fisicalistica}: categorizza il mondo
          secondo la fisica applicata.
    \item \textbf{Percettuale}: categorizza gli oggetti del mondo reale in base
          a come sono percepiti dall'utente come modello riconoscibile nello
          spazio o nel tempo.
    \item \textbf{Olistica}: si occupa dei processi che si svolgono nel tempo e
          del ruolo dei diversi partecipanti al processo. Uno di questi
          processi è
          chiamato semiosi. In una semiosi un interprete collega un oggetto ad
          un segno
          che gli dà significato.
\end{itemize}

Lo sviluppo e l'uso di EMMO è supportato dal Consiglio europeo per la
modellizzazione dei materiali (EMMC), che mira ad avvicinare la modellizzazione
dei materiali alle esigenze dell'industria. Uno dei principali obiettivi
dell'EMMC è di facilitare l'interoperabilità dei dati e dei modelli tra diversi
domini tecnici. Questo viene fatto attraverso l'istituzione di schemi di
metadati comuni e roadmap, che possono essere utilizzati come guida per le
ontologie di dominio EMMO.

\subsection{Caso d'uso}

\subsubsection{BattINFO}
BattINFO \cite{emmo} è un'ontologia di dominio per le celle batteria, i
componenti, i materiali e le loro interfacce, sviluppata per supportare
l'interoperabilità dei dati e i flussi di lavoro di intelligenza artificiale.
BattINFO si basa sull'ontologia EMMO e include un'ontologia di elettrochimica
come base fondamentale. Le classi di quest'ontologia si basano sulle
raccomandazioni
dell'International Union of Pure and Applied Chemistry (IUPAC) e della
International Electrotechnical Commission (IEC). L'ontologia fornisce anche
annotazioni per definire le etichette preferite per l'entità "Battery", oltre a
collegamenti alle voci di Wikipedia e dbpedia.

\section{Problematiche}
La problematica principale dell'esistenza di tutte queste ontologie è che non
sono correlate le une con le altre.

Le ontologie sopra elencate non sono allineate con SAREF, ciò comporta la
limitazione della riusabilità all'interno delle iniziative basate su SAREF e
viceversa.

Sarebbe consono a riguardo scegliere di adottare uno standard comune a tutti i
progetti inerenti a questi ambiti, in maniera tale da favorirne la comprensione
e il riutilizzo, nonchè la manutenibilità nel tempo.