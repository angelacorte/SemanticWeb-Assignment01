\chapter{Ontologie per Energy Storage System}

L'obiettivo principale delle ontologie per \textit{Energy Storage System} (ESS)
è di fornire una descrizione
semantica standardizzata dei dati relativi agli ESS, in modo da consentire
l'interoperabilità tra i sistemi di gestione dell'energia e i dispositivi ESS
provenienti da diversi fornitori. Questa interoperabilità è essenziale per
garantire l'efficienza, l'affidabilità e la sicurezza delle reti elettriche
intelligenti (smart grid) e per consentire l'integrazione di fonti di energia
rinnovabile.

Le ontologie ESS comprendono concetti come la capacità di stoccaggio
dell'energia, la capacità di erogazione di potenza, la capacità di ricarica, il
livello di sicurezza e
molti altri. Questi concetti sono organizzati in una struttura gerarchica che
definisce le relazioni tra di essi.

Esistono molte ontologie che coprono questi aspetti e ne estendono i concetti,
tra cui: \textit{OntoPowSys}, \textit{Energy Storage Systems Ontology},
\textit{Open Energy Ontology} e l'ecosistema basato su \textit{EMMO}.

\section{OntoPowSys}
OntoPowSys è un'ontologia del dominio dei sistemi di alimentazione, che è stata
sviluppata con lo scopo di supportare l'analisi e la progettazione di sistemi
di alimentazione elettrica, in particolare per applicazioni industriali e di
produzione di energia rinnovabile.

L'ontologia è stata sviluppata in OWL (Web Ontology Language) ed è stata
progettata per essere modulare e scalabile, in modo da poter essere utilizzata
in diverse applicazioni diagnostico/prognostiche.

L'ontologia OntoPowSys si basa su una struttura a tre livelli, che comprende:
la descrizione dei componenti, la descrizione delle funzioni e la descrizione
delle proprietà.

L'ontologia è stata utilizzata per sviluppare strumenti di supporto alla
decisione per l'analisi e la progettazione dei sistemi di alimentazione
elettrica, ad esempio per la selezione e la configurazione dei componenti del
sistema. Inoltre, OntoPowSys è stato integrato con altre ontologie del dominio
dell'energia, come la Open Energy Ontology, per creare una visione più completa
e integrata del sistema energetico.

OntoPowSys estende la gerarchia di classi e proprietà definita nel modulo di
sistema di alimentazione elettrica di OntoEIP. Estende
i concetti di supporto creando un framework per l'esecuzione di modelli
matematici in JPS (Joint Processing System è un sistema informatico che
utilizza ontologie per integrare conoscenze provenienti da diverse fonti e
consentire l'esecuzione di studi e simulazioni).

\subsection{Casi d'uso}

\subsubsection{Optimal Power Flow}

\subsubsection{Bio Diesel Plant Simulation}

\section{Energy Storage Systems ontology}

\section{EMMO-based Ecosystem}

\section{Problematiche}