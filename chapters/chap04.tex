\chapter{Ontologie per Energy Storage System}

L'obiettivo principale delle ontologie per \textit{Energy Storage System} (ESS)
è di fornire una descrizione
semantica standardizzata dei dati relativi agli ESS, in modo da consentire
l'interoperabilità tra i sistemi di gestione dell'energia e i dispositivi ESS
provenienti da diversi fornitori. Questa interoperabilità è essenziale per
garantire l'efficienza, l'affidabilità e la sicurezza delle reti elettriche
intelligenti (smart grid) e per consentire l'integrazione di fonti di energia
rinnovabile.

Le ontologie ESS comprendono concetti come la capacità di stoccaggio
dell'energia, la capacità di erogazione di potenza, la capacità di ricarica, il
livello di sicurezza e
molti altri. Questi concetti sono organizzati in una struttura gerarchica che
definisce le relazioni tra di essi.

Esistono molte ontologie che coprono questi aspetti e ne estendono i concetti,
tra cui: \textit{OntoPowSys}, \textit{Energy Storage Systems Ontology},
\textit{Open Energy Ontology} e l'ecosistema basato su \textit{EMMO}.

\section{OntoPowSys}
OntoPowSys è un'ontologia del dominio dei sistemi di alimentazione, che è stata
sviluppata con lo scopo di supportare l'analisi e la progettazione di sistemi
di alimentazione elettrica, in particolare per applicazioni industriali e di
produzione di energia rinnovabile.

L'ontologia è stata sviluppata in OWL (Web Ontology Language) ed è stata
progettata per essere modulare e scalabile, in modo da poter essere utilizzata
in diverse applicazioni diagnostico/prognostiche.

L'ontologia OntoPowSys si basa su una struttura a tre livelli, che comprende:
la descrizione dei componenti, la descrizione delle funzioni e la descrizione
delle proprietà.

L'ontologia è stata utilizzata per sviluppare strumenti di supporto alla
decisione per l'analisi e la progettazione dei sistemi di alimentazione
elettrica, ad esempio per la selezione e la configurazione dei componenti del
sistema. Inoltre, OntoPowSys è stato integrato con altre ontologie del dominio
dell'energia, come la Open Energy Ontology, per creare una visione più completa
e integrata del sistema energetico.

OntoPowSys estende la gerarchia di classi e proprietà definita nel modulo di
sistema di alimentazione elettrica di OntoEIP. Estende
i concetti di supporto creando un framework per l'esecuzione di modelli
matematici in JPS (Joint Processing System è un sistema informatico che
utilizza ontologie per integrare conoscenze provenienti da diverse fonti e
consentire l'esecuzione di studi e simulazioni).

\subsection{Casi d'uso}
In seguito verrà presentato qualche esempio di applicazione dell'ontologia
OntoPowSys.

\subsubsection{Optimal Power Flow}
OntoPowSys è stata integrata nell'agente Optimal Power Flow (OPF) della
piattaforma JPS, che consente di risolvere il problema dell'ottimizzazione
della distribuzione di energia elettrica. Esso utilizza l'ontologia
OntoPowSys per acquisire conoscenze sul sistema elettrico e fornire soluzioni
ottimali al problema dell'OPF.

Le informazioni vengono acquisite tramite query
SPARQL sul grafo di conoscenza JPS e utilizzate per eseguire i modelli
matematici. I risultati vengono aggiornati nel grafo di conoscenza e
visualizzati tramite strumenti appositi.

\subsubsection{Bio Diesel Plant Simulation}
Nell'ambito della simulazione di una bio-raffineria, OntoPowSys viene
utilizzata per la modellizzazione delle proprietà dei materiali coinvolti nei
processi di produzione del biodiesel. Questi modelli vengono integrati nella
piattaforma JPS, che utilizza web services e agenti specifici per eseguire
simulazioni e analisi.

In particolare, un agente di JPS viene utilizzato per lo
studio delle interazioni tra le varie fasi del processo di produzione del
biodiesel. Le informazioni raccolte dalla simulazione vengono utilizzate per
aggiornare la conoscenza presente nella piattaforma JPS e per generare
risultati visualizzabili tramite appositi strumenti.

\section{Energy Storage Systems Ontology}
L'Energy Storage Systems Ontology \cite{EnergyStorageSystemOntology} è
un'ontologia, che dispone di 10 classi, sviluppata per
rappresentare le conoscenze e le informazioni relative ai sistemi di accumulo
dell'energia (ESS). Questa ontologia è un modulo del “Domain Analysis-Based
Global Energy ontology” e si occupa di rappresentare dispositivi di accumulo di
energia, come batterie o condensatori.

\section{EMMO-based Ecosystem}
EMMO (European Materials and Modelling Ontology) è un'ontologia sviluppata
nell'ambito dell'omonimo progetto europeo, per la modellizzazione e la
descrizione dei materiali e dei fenomeni fisici che ne governano il
comportamento.

EMMO-based ecosystem si riferisce all'ecosistema di ontologie che si basano su
EMMO e che lo estendono per applicazioni specifiche. In particolare,
l'obiettivo di EMMO-based ecosystem è quello di fornire un insieme di ontologie
comuni e interoperabili per la descrizione di concetti e fenomeni nel dominio
dei materiali e delle scienze dei materiali. Questo è importante perché, data
la complessità dei materiali e dei fenomeni fisici coinvolti, è necessario
avere una rappresentazione univoca e standardizzata per garantire una corretta
comunicazione tra i ricercatori, per condividere e riutilizzare dati e
informazioni, e per supportare lo sviluppo di nuovi strumenti e tecnologie.

L'EMMO-based ecosystem comprende una serie di ontologie specifiche che
estendono EMMO, come ad esempio Thermodynamics Ontology per la descrizione dei
fenomeni termodinamici, Electrochemistry Ontology per la descrizione dei
fenomeni elettrochimici, e tante altre. Queste ontologie specifiche sono state
sviluppate per soddisfare le esigenze dei diversi sottodomini all'interno del
dominio dei materiali e delle scienze dei materiali. In questo modo, EMMO-based
ecosystem offre un insieme completo di ontologie per rappresentare i concetti e
i fenomeni nel dominio dei materiali in modo standardizzato e interoperabile.

\section{Problematiche}