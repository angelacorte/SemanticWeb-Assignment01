\chapter{Introduzione}
% \label{chap:introduction}
%----------------------------------------------------------------------------------------

Nel campo dell’IoT è essenziale scambiare e utilizzare dati e informazioni in maniera chiara e senza ambiguità.
per garantire chiarezza e disambiguità è necessario disporre di un modello che regoli non solo lo scambio di informazioni ma anche il loro significato. Solo in questo modo possiamo essere certi che i dati trasmessi siano compresi e utilizzati in modo corretto ed efficace, sia lato mittente che lato ricevente.
A tal proposito, il modello ontologico SAREF (Smart Appliance REFerence) contribuisce a risolvere i problemi di semantica ambigua e comprensione dei dati. 
SAREF è stato sviluppato per definire un vocabolario standardizzato di concetti e relazioni che rappresentano dispositivi e servizi presenti in una rete IoT, garantendo una maggiore interoperabilità tra i dispositivi e tra i vari servizi che si basano su di essi.
