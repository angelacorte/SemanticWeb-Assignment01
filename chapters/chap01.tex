\chapter{Introduzione}
% \label{chap:introduction}
%----------------------------------------------------------------------------------------

La gestione dell'energia è una delle sfide più rilevanti e complesse per la nostra società, con impatti significativi sull'economia, sull'ambiente e sulla qualità della vita delle persone. Per affrontare questa sfida nel contesto del web semantico ci si basa sulla standardizzazione dei dati e si utilizzano delle ontologie appartenenti al dominio energetico.

In particolare, \textit{SAREF} è un’ontologia sviluppata per supportare l’interoperabilità semantica tra diversi attori e servizi, la quale è stata estesa in versioni specifiche per le città (\textit{SAREF4City}), per gli edifici (\textit{SAREF4Building}) e per l'energia (\textit{SAREF4ENER}), al fine di adattare l'ontologia alle esigenze specifiche di questi contesti.

Tuttavia, per sfruttare appieno il potenziale di SAREF e delle sue estensioni, è necessario adottare strumenti e tecnologie che consentano l'interoperabilità e la condivisione dei dati in modo efficiente e affidabile. In questo contesto, il progetto \textit{InterConnect} offre un framework per l'interoperabilità semantica che permette di integrare dati provenienti da diverse fonti e di facilitare la condivisione delle conoscenze tra i vari attori del settore energetico.

Infine esistono delle ontologie specifiche del settore energetico che coprono tutti gli aspetti, dalla produzione alla distribuzione, all'uso e alla gestione dell'energia. Tra queste ci sono: \textit{OntoPowSys}, \textit{Energy Storage Systems Ontology},
\textit{Open Energy Ontology} e l'ecosistema basato su \textit{EMMO}.
L'utilizzo di queste ontologie rappresenta un importante passo avanti verso una gestione più efficiente, sostenibile e intelligente dell'energia.

